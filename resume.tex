%%%%%%%%%%%%%%%%%%%%%%%%%%%%%%%%%%%%%%%
% Deedy - One Page Two Column Resume
% LaTeX Template
% Version 1.1 (30/4/2014)
%
% Original author:
% Ayush Raj (http://debarghyadas.com)
%
% Original repository:
% https://github.com/deedydas/Deedy-Resume
%
% IMPORTANT: THIS TEMPLATE NEEDS TO BE COMPILED WITH XeLaTeX
%
% This template uses several fonts not included with Windows/Linux by
% default. If you get compilation errors saying a font is missing, find the line
% on which the font is used and either change it to a font included with your
% operating system or comment the line out to use the default font.
% 
%%%%%%%%%%%%%%%%%%%%%%%%%%%%%%%%%%%%%%
% 
% TODO:
% 1. Integrate biber/bibtex for article citation under publications.
% 2. Figure out a smoother way for the document to flow onto the next page.
% 3. Add styling information for a "Projects/Hacks" section.
% 4. Add location/address information
% 5. Merge OpenFont and MacFonts as a single sty with options.
% 
%%%%%%%%%%%%%%%%%%%%%%%%%%%%%%%%%%%%%%
%
% CHANGELOG:
% v1.1:
% 1. Fixed several compilation bugs with \renewcommand
% 2. Got Open-source fonts (Windows/Linux support)
% 3. Added Last Updated
% 4. Move Title styling into .sty
% 5. Commented .sty file.
%
%%%%%%%%%%%%%%%%%%%%%%%%%%%%%%%%%%%%%%%
%
% Known Issues:
% 1. Overflows onto second page if any column's contents are more than the
% vertical limit
% 2. Hacky space on the first bullet point on the second column.
%
%%%%%%%%%%%%%%%%%%%%%%%%%%%%%%%%%%%%%%

\documentclass[]{deedy-resume-openfont}


\begin{document}

%%%%%%%%%%%%%%%%%%%%%%%%%%%%%%%%%%%%%%
%
%     LAST UPDATED DATE
%
%%%%%%%%%%%%%%%%%%%%%%%%%%%%%%%%%%%%%%
\lastupdated

%%%%%%%%%%%%%%%%%%%%%%%%%%%%%%%%%%%%%%
%
%     TITLE NAME
%
%%%%%%%%%%%%%%%%%%%%%%%%%%%%%%%%%%%%%%


\namesection{Ayush}{Raj}{ \urlstyle{same}\url{https://llucifer97.github.io/} \\
\href{mailto:ayushraj.bit17@gmail.com}{ayushraj.bit17@gmail.com} 
}

%%%%%%%%%%%%%%%%%%%%%%%%%%%%%%%%%%%%%%
%
%     COLUMN ONE
%
%%%%%%%%%%%%%%%%%%%%%%%%%%%%%%%%%%%%%%

\begin{minipage}[t]{0.33\textwidth} 

%%%%%%%%%%%%%%%%%%%%%%%%%%%%%%%%%%%%%%
%     EDUCATION
%%%%%%%%%%%%%%%%%%%%%%%%%%%%%%%%%%%%%%

\section{Education} 

\descript{Birla Institute Of Technology, MESRA (patna campus)}
\descript{BE in Electronics and communication engg.}
\location{Expected MAY 2021 | Dhanbad,India }
\sectionsep


%%%%%%%%%%%%%%%%%%%%%%%%%%%%%%%%%%%%%%
%     LINKS
%%%%%%%%%%%%%%%%%%%%%%%%%%%%%%%%%%%%%%

\section{Links} 
Github:// \href{https://github.com/llucifer97}{\custombold{llucifer97}} \\
LinkedIn://  \href{https://www.linkedin.com/in/ayush-raj97/}{\custombold{ayushraj97}} \\
Codeforces://  \href{https://codeforces.com/profile/ayush_raj97}{\custombold{ayushraj97}} \\
Leetcode://  \href{https://leetcode.com/llucifer97/}{\custombold{ayushR7}}
\sectionsep

%%%%%%%%%%%%%%%%%%%%%%%%%%%%%%%%%%%%%%
%     COURSEWORK
%%%%%%%%%%%%%%%%%%%%%%%%%%%%%%%%%%%%%%

%%%%%%%%%%%%%%%%%%%%%%%%%%%%%%%%%%%%%%
%     SKILLS
%%%%%%%%%%%%%%%%%%%%%%%%%%%%%%%%%%%%%%

\section{Skills}
\runsubsection{Programming}\\
%\subsection{Programming}
\location{Experienced:}
c++ \textbullet{}  Algorithms \textbullet{} Datastructure\ \\ 
\location{Proficient:}
HTML \textbullet{} MongoDB \textbullet{} Machine Learning
Express.js \textbullet{} Nodejs\textbullet{} Reactjs\textbullet{} Python
\location{Familiar:}
Android \textbullet{} Django \textbullet{} Javascript \textbullet{} jquery
Linux  \textbullet{} Git \textbullet{} ReactNative
\sectionsep


\section{COURSEWORK}
\textbullet{}Fundamentals of Data Structures\\
\textbullet{}Fundamentals of Unix and C
 Programming \\
\textbullet{}Computer Networking\\
\textbullet{}Data communication \\
\textbullet{}Database Management Systems
%\subsection{Programming}

%%%%%%%%%%%%%%%%%%%%%%%%%%%%%%%%%%%%%%
%
%     COLUMN TWO
%
%%%%%%%%%%%%%%%%%%%%%%%%%%%%%%%%%%%%%%

\end{minipage} 
\hfill
\begin{minipage}[t]{0.66\textwidth} 

%%%%%%%%%%%%%%%%%%%%%%%%%%%%%%%%%%%%%%
%     EXPERIENCE
%%%%%%%%%%%%%%%%%%%%%%%%%%%%%%%%%%%%%%

\section{Experience}

\runsubsection{Redhenlab}
\descript{| Google Summer Of Code’19 Student}
\location{Expected May 2019 – Aug 2019 | Remote}
\vspace{\topsep} % Hacky fix for awkward extra vertical space
\begin{tightemize}\item Planned,designed and developed a Gesture Annotator Tool to visualize and
validate human-pose extracted from ancient paintings under the mentorship of Leo Impett, with over 2000 lines of the codebase.
\item Developed the UI and integrated the plotting features with Backend API and deployed in singularity container.
\item Wrote web-crawler to collect GBs of data from online sources and bash scripts
to automate the pipeline.
\item Project can be found  \href{https://summerofcode.withgoogle.com/archive/2019/projects/5565179075493888/}{Here}.
\end{tightemize}
\sectionsep

\sectionsep

%%%%%%%%%%%%%%%%%%%%%%%%%%%%%%%%%%%%%%
%     Project
%%%%%%%%%%%%%%%%%%%%%%%%%%%%%%%%%%%%%%

\section{PROJECTS}
\runsubsection{VIPANI}
\descript{| \href{https://github.com/llucifer97/Vipani}{code} | FULLSTACK}
\location{June 2020 - present }
Vipani is an online e-commerce web app to serve its user’s authentic ayurvedic
medicine across the globe
\begin{tightemize}
\item Build a Restful API and created endpoints to perform CRUD operations.
\item Implemented security feature, user authentication with different strategies.
\item Designed and implemented database schema in MongoDB .
\end{tightemize}
\sectionsep

\runsubsection{GEEKCHAT}
\descript{| \href{https://geeky-chat.herokuapp.com}{Demo} | FULLSTACK}
\location{July 2020 }
GeekChat is an realtime multiuser chat app,an online place for geeks to share their
ideas.
\begin{tightemize}
\item Provides a Global Chat which can be used by anyone using the application to
broadcast messages to everyone else in the particular room.
\item Updates conversation lists, user lists and messages in real time.
\item Socket.io is used ,to enable real-time, bidirectional communication
\end{tightemize}
\sectionsep

\runsubsection{QUORA-QUESTION}
\descript{| \href{https://github.com/llucifer97/quora-question-matching}{code} | Machine Learning}
\location{ January 2020}
Implemented a classifier to recognize question pairs asked on quora using machine
learning algorithms.
\begin{tightemize}
\item Used Natural language processing to extract advanced features to increase the
accuracy of the model.
\item Final model delivered a precision of 86 per cent with 87 per cent recall.
\end{tightemize}
\sectionsep


\runsubsection{FACEBOOK RECOMMENDATION}
\descript{| \href{https://github.com/llucifer97/facebook_case_study}{code} | Machine Learning}
\location{ March 2019-April 2019}
Investigated the FB user data and proposed different visualization techniques to
interpret it as a graph-based problem and advanced with graph mining technique to
identify the correlation between users and predicted missing links to recommend
users
\begin{tightemize}
\item Designed the solution by supervised learning and derived features using
visualisation and graph algorithms .
\item The model achieved the train F1 score of 92 percent after parameter tuning
and further code optimization .
\end{tightemize}
\sectionsep


%%%%%%%%%%%%%%%%%%%%%%%%%%%%%%%%%%%%%%
%     AWARDS
%%%%%%%%%%%%%%%%%%%%%%%%%%%%%%%%%%%%%%

\section{ACHIEVEMENTS} 
\begin{tabular}{rll}
2018	    & Got third position in Machine Learning Hacakthon(BIT MESRA)\\

\end{tabular}
\sectionsep

%%%%%%%%%%%%%%%%%%%%%%%%%%%%%%%%%%%%%%
%     SOCIETIES
%%%%%%%%%%%%%%%%%%%%%%%%%%%%%%%%%%%%%%


\end{minipage} 
\end{document}
